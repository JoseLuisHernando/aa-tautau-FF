%% Main-Version Draft
\documentclass[12pt,a4]{article}
%% -------------------------------
%% |          Packages           |
%% -------------------------------
 \usepackage{amsmath}
 \usepackage{mathtools} 
 \usepackage{graphicx}
 \usepackage[merge,numbers,compress]{natbib}
 \usepackage[T1]{fontenc}
 \usepackage{booktabs}
 \usepackage{xcolor} 
 \usepackage{xspace}
 \usepackage{dcolumn}
 \usepackage{placeins}
 \usepackage[colorlinks=true,citecolor=blue!50!black,linkcolor=black]{hyperref}
%  \usepackage[pdfborder={0 0 0}]{hyperref}
 \usepackage{caption}
 \usepackage
 [subrefformat=parens,position=top,skip=-15pt,margin=15pt,justification=justified,singlelinecheck=false]
 {subcaption}
 \usepackage{multirow}
\usepackage{slashed}
\usepackage[bottom]{footmisc}
\usepackage{comment}
\usepackage{bigints}
\usepackage{hhline}
\usepackage[normalem]{ulem}

% for the check and cross symbols
\usepackage{pifont}
\newcommand{\cmark}{\ding{51}}
\newcommand{\xmark}{\ding{55}}


%\oddsidemargin 00pt \evensidemargin 00pt
%\topmargin 00pt \headheight 00pt \headsep 00pt
%%\footheight 12pt \footskip 30pt
%\textheight 232mm \textwidth 160mm

\setlength{\evensidemargin}{00pt}
\setlength{\oddsidemargin}{00pt}
\setlength{\topmargin}{0.00pt}
\setlength{\textwidth}{160mm}
\setlength{\textheight}{232mm}
\setlength{\headheight}{00pt}
\setlength{\headsep}{00pt}
\setlength{\voffset}{00pt}
\setlength{\hoffset}{00pt}
%\setlength{\paperheight}{29.7cm}

\renewcommand{\topfraction}{1.0}
\renewcommand{\bottomfraction}{1.0}
\renewcommand{\textfraction}{0.15}
\renewcommand{\floatpagefraction}{0.8}

\newcommand{\SD}[1]{{ {\color{orange}{ [SD: #1]}} }}
\newcommand{\MP}[1]{{ {\color{blue}{ [MP: #1]}} }}
\newcommand{\JH}[1]{{ {\color{violet}{ [JH: #1]}} }}
\newcommand{\change}[1]{{{\color{red}{#1}}}}
\newcommand{\new}[1]{{{\color{olive}{#1}}}}

\newcommand{\processmue}{\gamma\gamma \to \tau^+ \tau^- \to e^+ \mu^- \bar{\nu}_\tau \nu_\tau \bar{\nu}_{\mu} \nu_{e}}

\newcommand{\process}{\gamma\gamma \to \tau^+ \tau^- \to \ell'^+ \ell^- \bar{\nu}_\tau \nu_\tau \bar{\nu}_{\ell} \nu_{\ell'}}

\newcommand{\M}{\mathcal{M}}

%% ---------------------------------
%% | ToDo Marker - only for draft! |
%% ---------------------------------
% Remove this section for final version!
% \setlength{\marginparwidth}{20mm}
% 
% \newcommand{\margtodo}
% {\marginpar{\textbf{\textcolor{red}{ToDo}}}{}}
% 
% \newcommand{\todo}[1]
% {{\textbf{\textcolor{red}{(\margtodo{}#1)}}}{}}

\newcommand{\eq}[1]{\begin{equation} #1 \end{equation}}
\newcommand{\eqa}[1]{\begin{eqnarray} #1 \end{eqnarray}}

\newcommand{\Pl}{\ell}
\newcommand{\fb}{{\ensuremath\unskip\,\text{fb}}\xspace}

% % new commands for cross referencing
\def\refeq#1{\mbox{(\ref{#1})}}
\def\reffi#1{\mbox{Figure~\ref{#1}}}
\def\reffis#1{\mbox{Figures~\ref{#1}}}
\def\refta#1{\mbox{Table~\ref{#1}}}
\def\reftas#1{\mbox{Tables~\ref{#1}}}
\def\refse#1{\mbox{Section~\ref{#1}}}
\def\refapp#1{\mbox{App.~\ref{#1}}}
\def\citere#1{\mbox{Ref.~\cite{#1}}}
\def\citeres#1{\mbox{Refs.~\cite{#1}}}

\def\EXP#1{{\ensuremath \times 10^{#1}}}
\newcommand{\LO}{\ensuremath{\text{LO}}}
\newcommand{\NLO}{\ensuremath{\text{NLO}}}

\newcommand{\ri}{\mathrm i}
\newcommand{\rd}{\mathrm d}

\newcommand{\ie}{\emph{i.e.}\ }
\newcommand{\eg}{\emph{e.g.}\ }

% define punctuation-aware footnote macro
\let\origfootnote\footnote
\renewcommand{\footnote}[1]{\kern.06em\origfootnote{#1}}
\newcommand{\punctfootnote}[1]{\kern-.06em\origfootnote{#1}}

\def\be{\begin{equation}}
\def\ee{\end{equation}}

\newcommand{\qqb}{\ensuremath{q\bar{q}}\xspace}
\newcommand{\PH}{\ensuremath{\text{H}}\xspace}
\newcommand{\Pj}{\ensuremath{\text{j}}\xspace}
\newcommand{\Pp}{\ensuremath{\text{p}}\xspace}
\newcommand{\Pe}{\ensuremath{\text{e}}\xspace}
\newcommand{\Pb}{\ensuremath{\text{b}}\xspace}
\newcommand{\Pq}{\ensuremath{{q}}\xspace}
\newcommand{\Pt}{\ensuremath{\text{t}}\xspace}
\newcommand{\Pu}{\ensuremath{\text{u}}\xspace}
\newcommand{\Pd}{\ensuremath{\text{d}}\xspace}
\newcommand{\Ps}{\ensuremath{\text{s}}\xspace}
\newcommand{\Pc}{\ensuremath{\text{c}}\xspace}
\newcommand{\Pg}{\ensuremath{\text{g}}\xspace}
\newcommand{\Pw}{\ensuremath{\text{w}}\xspace}
\newcommand{\PW}{\ensuremath{\text{W}}\xspace}
\newcommand{\PZ}{\ensuremath{\text{Z}}\xspace}
\newcommand{\Pbj}{\ensuremath{\text{j_b}}\xspace}
                                    
\newcommand{\ME}{\ensuremath{m_\Pe}\xspace}
\newcommand{\MM}{\ensuremath{m_{\mu}}\xspace}
\newcommand{\ML}{\ensuremath{m_{\tau}}\xspace}
\newcommand{\Mt}{\ensuremath{m_\Pt}\xspace}
\newcommand{\MH}{\ensuremath{M_\PH}\xspace}
\newcommand{\MWOS}{\ensuremath{M_\PW^\text{OS}}\xspace}
\newcommand{\MW}{\ensuremath{M_\PW}\xspace}
\newcommand{\MZOS}{\ensuremath{M_\PZ^\text{OS}}\xspace}
\newcommand{\MZ}{\ensuremath{M_\PZ}\xspace}
\newcommand{\Mb}{\ensuremath{m_\Pb}\xspace}
\newcommand{\Mc}{\ensuremath{m_\Pc}\xspace}
\newcommand{\Gt}{\ensuremath{\Gamma_\Pt}\xspace}
\newcommand{\GH}{\ensuremath{\Gamma_\PH}\xspace}
\newcommand{\GZ}{\ensuremath{\Gamma_\PZ}\xspace}
\newcommand{\GZOS}{\ensuremath{\Gamma_\PZ^\text{OS}}\xspace}
\newcommand{\GW}{\ensuremath{\Gamma_\PW}\xspace}
\newcommand{\GWOS}{\ensuremath{\Gamma_\PW^\text{OS}}\xspace}

\newcommand{\keV}{\ensuremath{\,\text{keV}}\xspace}
\newcommand{\MeV}{\ensuremath{\,\text{MeV}}\xspace}
\newcommand{\GeV}{\ensuremath{\,\text{GeV}}\xspace}
\newcommand{\TeV}{\ensuremath{\,\text{TeV}}\xspace}

\newcommand{\alphas}{\ensuremath{\alpha_\text{s}}\xspace}
\newcommand{\order}[1]{\ensuremath{\mathcal{O}{\left(#1\right)}}\xspace}

\newcommand{\abs}[1]{\left|#1\right|}
\newcommand{\deltar}{\ensuremath{\Delta R}\xspace}

\newcommand{\GF}{\ensuremath{G_\mu}}

\newcommand{\pt}{\ensuremath{p_\text{T}}\xspace}
\newcommand{\ptsub}[1]{\ensuremath{p_{\text{T},#1}}\xspace}

\renewcommand{\Re}{\mathop{\mathrm{Re}}\nolimits}
\renewcommand{\Im}{\mathop{\mathrm{Im}}\nolimits}

\newcommand{\MVOS}{\ensuremath{M_{\text{V}}^\text{OS}}\xspace}%
\newcommand{\GVOS}{\ensuremath{\Gamma_{\text{V}}^\text{OS}}\xspace}%

\newcommand{\vh}{\ensuremath{\vphantom{\int_A^A}}}
\newcommand{\vH}{\ensuremath{\vphantom{\int_\int^\int}}}

\newcommand{\sq}{\tilde{q}}
\newcommand{\su}{\tilde{u}}
\newcommand{\sd}{\tilde{d}}
\newcommand{\gl}{\tilde{g}}
\def\bom#1{{\mbox{\boldmath $#1$}}}
\newcommand\nn         {\nonumber}
\newcommand{\sul}{\tilde{u}_L}
\newcommand{\scl}{\tilde{c}_L}
\newcommand{\sdl}{\tilde{d}_L}
\newcommand{\ssl}{\tilde{s}_L}
\newcommand{\sur}{\tilde{u}_R}
%\newcommand{\scr}{\tilde{c}_R}
\newcommand{\sdr}{\tilde{d}_R}
\newcommand{\ssr}{\tilde{s}_R}
\newcommand{\stone}{\tilde{t}_1}
\newcommand{\sbone}{\tilde{b}_1}
\newcommand{\sttwo}{\tilde{t}_2}
\newcommand{\sbtwo}{\tilde{b}_2}
\newcommand{\neutone}{\tilde{\chi}^0_1}
\newcommand\sss{\mathchoice%
{\displaystyle}%
{\scriptstyle}%
{\scriptscriptstyle}%
{\scriptscriptstyle}%
}
\newcommand{\newc}{\newcommand}
% \newc{\be}{\begin{equation}}
% \newc{\ee}{\end{equation}}
\newc{\bi}{\begin{itemize}}
\newc{\ei}{\end{itemize}}
\newc{\benu}{\begin{enumerate}}
\newc{\eenu}{\end{enumerate}}
\newc{\bc}{\begin{center}}
\newc{\ec}{\end{center}}
\newc{\bfig}{\begin{figure}}
\newc{\efig}{\end{figure}}
\newc{\qbar}{\bar{q}}
\newc{\go}{\tilde{g}}
\newc{\PB}{\textsc{Powheg-Box}}
\newcommand\matB{{\cal B}}
\newcommand\matR{{\cal R}}
\newcommand\matV{{\cal V}}
\newcommand\matO{{\cal O}}
\newcommand\matF{{\cal F}}

\newcommand{\MonteTau}{{\sc MonteTau}\xspace}
\newcommand{\gammaUPC}{{\sc gamma-UPC}\xspace}
\newcommand{\libepa}{{\sc libepa}\xspace}
\newcommand{\TAUOLA}{{\sc TAUOLA}\xspace}
\newcommand{\PHOTOS}{{\sc PHOTOS}\xspace}
\newcommand{\Herwig}{{\sc Herwig}\xspace}
\newcommand{\cpp}{C\texttt{++}\xspace}
\newcommand{\VEGAS}{VEGAS\xspace}
\newcommand{\gsl}{GNU Scientific Library\xspace}
\newcommand{\Mathematica}{{\sc Mathematica}\xspace}
\newcommand{\FeynArts}{{\sc FeynArts}\xspace}
\newcommand{\FormCalc}{{\sc FormCalc}\xspace}
\newcommand{\McMule}{{\sc McMule}\xspace}
\newcommand{\Recola}{{\sc Recola}\xspace}
\newcommand{\Sherpa}{{\sc Sherpa}\xspace}
\newcommand{\SherpaRecola}{{\sc Sherpa\!{+}Recola}\xspace}
\newcommand{\MoCaNLO}{{\sc MoCaNLO}\xspace}
\newcommand{\MoCaNLORecola}{{\sc  MoCaNLO{+}Recola}\xspace}
\newcommand{\Rivet}{{\sc Rivet}\xspace}
\newcommand{\Amegic}{A\protect\scalebox{0.8}{MEGIC}\xspace}
\newcommand{\Comix}{C\protect\scalebox{0.8}{OMIX}\xspace}
\newcommand{\OpenLoops}{O\protect\scalebox{0.8}{PEN}L\protect\scalebox{0.8}{OOPS}\xspace}
\newcommand{\Njet}{N\protect\scalebox{0.8}{JET}\xspace}
\newcommand{\BlackHat}{B\protect\scalebox{0.8}{LACK}H\protect\scalebox{0.8}{AT}\xspace}
\newcommand{\Gosam}{G\protect\scalebox{0.8}{O}S\protect\scalebox{0.8}{AM}\xspace}
\newcommand{\mocanlo}{{\sc MoCaNLO}\xspace}
\newcommand{\collier}{{\sc Collier}\xspace}
\newcommand{\CutTools}{{\sc CutTools}\xspace}
\newcommand{\madgraph}{{\sc\small MadGraph5\_aMC@NLO}\xspace}
\newcommand{\madgraphbis}{{\sc\small MG5\_aMC@NLO}\xspace}
\newcommand{\rT}{{\mathrm{T}}}
\newcolumntype{.}{D{.}{.}{-1}}
\newcolumntype{d}[1]{D{.}{.}{#1}}

\newcommand{\QED}{\ensuremath{\text{QED}}}
\newcommand{\QCD}{\ensuremath{\text{QCD}}}
\newcommand{\EW}{\ensuremath{\text{EW}}}
\newcommand{\EWapprox}{\ensuremath{\text{EW}_\text{approx}}}
\newcommand{\EWvirt}{\ensuremath{\text{EW}_\text{virt}}}
\newcommand{\QCDmEW}{{\ensuremath{\text{QCD--\EW}}}}
\newcommand{\QCDpEW}{{\ensuremath{\QCD+\EW}}}
\newcommand{\QCDtEW}{{\ensuremath{\QCD\times\EW}}}
\newcommand{\QCDpEWapprox}{{\ensuremath{\QCD+\EWapprox}}}
\newcommand{\QCDtEWapprox}{{\ensuremath{\QCD\times\EWapprox}}}
\newcommand{\bare}{{\ensuremath{\text{bare}}}}
\newcommand{\drs}{{\ensuremath{\text{drs}}}}

\newcommand{\MEPS}{\text{\textsc{MePs}}\xspace}
\newcommand{\MEPSatLO}{\text{\textsc{MePs@Lo}}\xspace}
\newcommand{\MEPSatNLO}{\text{\textsc{MePs@Nlo}}\xspace}
\newcommand{\MEPSatNLOQCD}{\MEPSatNLO\ \QCD\xspace}
\newcommand{\MEPSatNLOQCDpEWapprox}{\MEPSatNLO\ \ensuremath{\QCD+\EWapprox}\xspace}
\newcommand{\MEPSatNLOQCDtEWapprox}{\MEPSatNLO\ \ensuremath{\QCD\times\EWapprox}\xspace}
\newcommand{\MENLOPS}{\text{\textsc{MeNloPs}}\xspace}
\newcommand{\MENLOPSQCD}{\MENLOPS\ \QCD\xspace}
\newcommand{\MCatNLO}{\text{\textsc{Mc@Nlo}}\xspace}

\newcommand{\mr}[1]{\ensuremath{\mathrm{#1}}}
\newcommand{\mc}[1]{\ensuremath{\mathcal{#1}}}
\newcommand{\muR}{\ensuremath{\mu_{\mr{R}}}}
\newcommand{\muF}{\ensuremath{\mu_{\mr{F}}}}
\newcommand{\muQ}{\ensuremath{\mu_{\mr{Q}}}}
\newcommand{\muCKKW}{\ensuremath{\mu_{\text{CKKW}}}}
\newcommand{\done}{\ensuremath{\mr{d}}}
\newcommand{\Qcut}{\ensuremath{Q_\text{cut}}}
\newcommand{\nmax}{\ensuremath{n_\text{max}}}
\newcommand{\nmaxnlo}{\ensuremath{n_\text{max}^\text{NLO}}}
\newcommand{\mucore}{\ensuremath{\mu_\text{core}}}
\newcommand{\njet}{\ensuremath{n_\text{jet}}}
\newcommand{\Bbar}{\ensuremath{\overline{\mr{B}}}}
\newcommand{\deltaEWapprox}{\ensuremath{\delta_\EW^\text{approx}}}
\newcommand{\ETWmean}{\ensuremath{\overline{E}_{\mr{T},\PW}}}

%\renewcommand{\vec}[1]{\mathbf{#1}}
\colorlet{tableoverheadcolor}{gray!37.5}
\colorlet{tableheadcolor}{gray!25}
\colorlet{tablerowcolor}{gray!12.5}

\newcommand{\lsim}
{\;\raisebox{-.3em}{$\stackrel{\displaystyle <}{\sim}$}\;}
\newcommand{\gsim}
{\;\raisebox{-.3em}{$\stackrel{\displaystyle >}{\sim}$}\;}
\def\asymp#1{\;\raisebox{-.4em}{$\widetilde{\scriptstyle #1}$}\;}

\newlength{\width}
\newlength{\height}
\newcommand{\brabar}[1]{%
    \settoheight{\height}{\ensuremath{#1}}%
    \settowidth{\width}{\ensuremath{#1}}%
    \makebox[0pt][l]{\ensuremath{#1}}%
    \raisebox{1.26ex}{\scalebox{.3}{\textbf{(}}}%
    \rule[1.41\height]{0.7\width}{0.35pt}%
    \raisebox{1.26ex}{\scalebox{.3}{\textbf{)}}}%
}

% modifications for drafts for drafts
\newcommand{\mpar}[1]{{\marginpar{\hbadness10000%
                      \sloppy\hfuzz10pt\boldmath\bf\textcolor{red}{#1}}}%
                      \typeout{marginpar: #1}\ignorespaces}
\marginparwidth 1.2cm
\marginparsep 0.2cm
\def\draftdate{\relax}
\def\mda{\relax}
\def\mua{\relax}
\def\mla{\relax}
\def\draft{
\def\thtystars{******************************}
\def\sixtystars{\thtystars\thtystars}
\typeout{}
\typeout{\sixtystars**}
\typeout{* Draft mode!
         For final version remove \protect\draft\space in source file *}
\typeout{\sixtystars**}
\typeout{}
\def\draftdate{\today}
\def\mua{\marginpar[\boldmath\hfil$\uparrow$]%
                   {\boldmath$\uparrow$\hfil}\color{black}%
                    \typeout{marginpar: $\uparrow$}\ignorespaces}
\def\mda{\color{red}\marginpar[\boldmath\hfil$\downarrow$]%
                   {\boldmath$\downarrow$\hfil}%
                    \typeout{marginpar: $\downarrow$}\ignorespaces}
\def\mla{\marginpar[\boldmath\hfil$\rightarrow$]%
                   {\boldmath$\leftarrow $\hfil}%
                    \typeout{marginpar: $\leftrightarrow$}\ignorespaces}
\def\Mua{\marginpar[\boldmath\hfil$\Uparrow$]%
                   {\boldmath$\Uparrow$\hfil}\color{black}%
                    \typeout{marginpar: $\uparrow$}\ignorespaces}
\def\Mda{\color{red}\marginpar[\boldmath\hfil$\Downarrow$]%
                   {\boldmath$\Downarrow$\hfil}%
                    \typeout{marginpar: $\downarrow$}\ignorespaces}
\def\Mla{\marginpar[\boldmath\hfil\textcolor{red}{$\Rightarrow$}]%
                   {\boldmath\textcolor{red}{$\Leftarrow $}\hfil}%
                    \typeout{marginpar: $\leftrightarrow$}\ignorespaces}
\overfullrule 5pt
\oddsidemargin 15mm
\marginparwidth 29mm
}

\newcommand{\hl}{\vphantom{$\int_A^B$}}
\newcommand{\mhl}{\vphantom{\int_A^B}}
\newcommand{\mhhl}{\vphantom{\frac{\pi^2}{6}}}
\newcommand{\Hl}{\vphantom{$\int\limits_A^B$}}
\newcommand{\mHl}{\vphantom{\int\limits_A^B}}
\newcommand{\wm}{\phantom{$-$}}
\newcommand{\wn}{\phantom{$0$}}
\newcommand{\mwm}{\phantom{-}}

\newcolumntype{C}{>{\centering\arraybackslash}p{0.105\textwidth}}

% switch on draft mode
%\draft



\begin{document}

\title{\hfill ~\\[-30mm]
\phantom{h} \hfill\mbox{\small FR-PHENO-2026-XXX}
\\[1cm]
\vspace{13mm} 
%\textbf{Subtleties in the determination of the anomalous magnetic moment and the electric dipole moment of leptons via ultraperipheral heavy-ion collisions at the LHC.}
\textbf{Imprint of the anomalous magnetic moment 
%and the electric dipole moment 
of the $\tau$-lepton in the $\tau^+\tau^-$ production via ultraperipheral heavy-ion collisions at the LHC.}
}

\date{}
\author{
Stefan Dittmaier\footnote{E-mail:
  \texttt{stefan.dittmaier@physik.uni-freiburg.de}},
José Luis Hernando Ariza\footnote{E-mail: 
  \texttt{jose.luis.hernando@physik.uni-freiburg.de}},
Mathieu Pellen\footnote{E-mail:
  \texttt{mathieu.pellen@physik.uni-freiburg.de}}
\\[9mm]
{\small\it Universit\"at Freiburg, Physikalisches Institut,} \\
{\small\it Hermann-Herder-Str.\ 3, 79104 Freiburg, Germany}\\[3mm]
}
\maketitle

\begin{abstract}
\noindent


\end{abstract}
\thispagestyle{empty}


\newpage
\setcounter{page}{1}

\tableofcontents
% \newpage

\section{Introduction}

%- Should we include again the section about the equivalent-photon approximation(?)


\newpage
\section{Features of the calculation}
\label{sec:features}

\subsection{Equivalent-photon approximation}
\label{sec:EPA}

- Describe the EPA (as in the previous paper).

\subsection{Form-factor approach}
\label{sec:form_factor}

%- Include this section in Features of the calculation and add section on EPA (?).

- Keep this section for a general lepton and focus on the $\tau$-lepton in the numerics to emphasize that the same analysis can be done for $e$ and $\mu$ as a check of the method(?).

\subsubsection{Form-factor decomposition of the $\bf{\gamma}\bf{\bar{\tau}}\bf{\tau}$ vertex function}
\label{sec:vertex}

- Mention the problem of the off-shell $\tau$ in the standard approach.

- Outline the way how the general form-factor decomposition is obtained.

- Give the general form-factor decomposition in the on-shell basis.

\begin{comment}
\begin{align} \label{eq:vertex_on-shell}
  \Gamma_\mu^{\gamma\bar{f}f}(k,\bar{p},p) = 
  e\bigg[&
  \gamma_\mu \,g_{1}
  + \gamma_\mu\gamma_5 \,g_{2}
  + \frac{\ri\sigma_{\mu\nu}k^\nu}{2m} \,g_{3}
  + \frac{\sigma_{\mu\nu}\gamma_5k^\nu}{2m} \,g_{4} 
  + \frac{\ri k_\mu}{2m}  \,g_{5}
  + \frac{k_\mu}{2m} \gamma_5 \,g_{6} 
  \mhl \notag \\
  &+ \big(\Lambda_+(\bar{p})\gamma_\mu + \gamma_\mu\Lambda_-(p)\big)\,g_{7} 
  +  \big(\Lambda_+(\bar{p})\gamma_\mu - \gamma_\mu\Lambda_-(p)\big)\,g_{8} 
  \mhl \notag \\
  &+ \big(\Lambda_+(\bar{p})\gamma_\mu\gamma_5 + \gamma_\mu\gamma_5\Lambda_-(p)\big)\,g_{9} 
  +  \big(\Lambda_+(\bar{p})\gamma_\mu\gamma_5 - \gamma_\mu\gamma_5\Lambda_-(p)\big)\,g_{10}
  \mhl \notag \\
  &+ \frac{\bar{k}_\mu}{2m}\big(\Lambda_+(\bar{p}) + \Lambda_-(p)\big)\,g_{11}
  +  \frac{\bar{k}_\mu}{2m}\big(\Lambda_+(\bar{p}) - \Lambda_-(p)\big)\,g_{12} 
  \mhl \notag \\
  &+ \frac{\ri\bar{k}_\mu}{2m}\big(\Lambda_+(\bar{p})\gamma_5 + \gamma_5\Lambda_-(p)\big)\,g_{13} 
  + \frac{\ri \bar{k}_\mu}{2m}\big(\Lambda_+(\bar{p})\gamma_5 - \gamma_5\Lambda_-(p)\big)\,g_{14} 
  \mhl \notag \\
  &+ \frac{\ri k_\mu}{2m}\big(\Lambda_+(\bar{p}) + \Lambda_-(p) \big)\,g_{15}
  + \frac{\ri k_\mu}{2m}\big(\Lambda_+(\bar{p}) - \Lambda_-(p) \big)\,g_{16}
  \mhl\notag \\
  &+ \frac{k_\mu}{2m}\big(\Lambda_+(\bar{p})\gamma_5 + \gamma_5\Lambda_-(p)\big)\,g_{17}  
  + \frac{k_\mu}{2m}\big(\Lambda_+(\bar{p})\gamma_5 - \gamma_5\Lambda_-(p)\big)\,g_{18}  
  \mhl\notag \\
  &+ \Lambda_+(\bar{p})\gamma_\mu\Lambda_-(p) \,g_{19}
  + \Lambda_+(\bar{p})\gamma_\mu\gamma_5\Lambda_-(p) \,g_{20}
  \mhl\notag \\
  &+ \frac{\bar{k}_\mu}{2m}\Lambda_+(\bar{p})\Lambda_-(p) \,g_{21}
  + \frac{\ri \bar{k}_\mu}{2m}\Lambda_+(\bar{p})\gamma_5\Lambda_-(p) \,g_{22}
  \mhl\notag \\
  &+ \frac{\ri k_\mu}{2m}\Lambda_+(\bar{p})\Lambda_-(p) \,g_{23}
   + \frac{k_\mu}{2m}\Lambda_+(\bar{p})\gamma_5\Lambda_-(p) \,g_{24}\bigg]
  \mhl,
\end{align}

- Give the transformation properties under space-time discrete symmetries of each term (no details). 

Parity check: $g_1$, $g_3$, $g_5$, $g_7$, $g_8$, $g_{11}$, $g_{12}$, $g_{15}$, $g_{16}$,
$g_{19}$, $g_{21}$, $g_{22}$.

Parity cross: $g_2$, $g_4$, $g_6$, $g_9$, $g_{10}$, $g_{13}$, $g_{14}$, $g_{17}$, $g_{18}$,
$g_{20}$, $g_{22}$, $g_{24}$.

Charge conjugation check: $g_1$, $g_3$, $g_4$, $g_7$, $g_{10}$, $g_{11}$, $g_{13}$, $g_{16}$, $g_{18}$, $g_{19}$, $g_{21}$, $g_{22}$ 

Charge conjugation cross: $g_2$, $g_5$, $g_6$, $g_8$, $g_9$, $g_{12}$, $g_{14}$, $g_{15}$, $g_{17}$, $g_{20}$, $g_{24}$


Time reversal check:

Time reversal cross: 

\begin{table}[h!]
\centering{
\begin{tabular}{l|c|c|c|c|c}
   \hl Space--time symmetry & \wn$\mc{P}$\wn & \wn$\mc{C}$\wn &\wn$\mc{T}$\wn & \;$\mc{CP}$\; & \,$\mc{CPT}$\, \\
   \hline\hline
   \hl $g_1,g_3,g_7,g_{9},g_{13},g_{15},g_{19},g_{21}$ & \cmark   & \cmark   & \cmark   & \cmark    & \cmark     \\      
   \hl $g_2,g_8,g_{14},g_{20}$                   & \xmark   & \xmark   & \cmark   & \cmark    & \cmark     \\      
   \hl $g_4,g_{10},g_{16},g_{22}$                & \xmark   & \cmark   & \xmark   & \xmark    & \cmark     \\      
   \hl $g_5,g_{11},g_{17},g_{23}$                & \cmark   & \xmark   & \xmark   & \xmark    & \cmark     \\      
   \hl $g_6,g_{12},g_{18},g_{24}$                & \xmark   & \xmark   & \cmark   & \cmark    & \cmark     \\      
\end{tabular}
}
\caption{Summery of the preserved/violated space--time symmetries by the covariant structures associated with each form factor $g_i$.
The symbols \cmark\, and \xmark\, indicate if the space--time symmetry is preserved or violated, respectively.}
\end{table}
\end{comment}

\JH{Change the structures to $\Lambda_+(\bar{p}) \pm \Lambda_-(p)$ and adapt table.}

\begin{align} \label{eq:vertex_on-shell}
  \Gamma_\mu^{\gamma\bar{f}f}(k,\bar{p},p) = 
  e\bigg[&
  \gamma_\mu \,g_{1}
  + \gamma_\mu\gamma_5 \,g_{2}
  + \frac{\ri\sigma_{\mu\nu}k^\nu}{2m} \,g_{3}
  + \frac{\sigma_{\mu\nu}\gamma_5k^\nu}{2m} \,g_{4} 
  + \frac{\ri k_\mu}{2m}  \,g_{5}
  + \frac{k_\mu}{2m} \gamma_5 \,g_{6} 
  \notag \\
  &+ \Lambda_+(\bar{p})\gamma_\mu \,g_{7} 
  + \Lambda_+(\bar{p})\gamma_\mu\gamma_5 \,g_{8} 
  + \frac{\bar{k}_\mu}{2m}\Lambda_+(\bar{p}) \,g_{9}
  + \frac{\ri \bar{k}_\mu}{2m}\Lambda_+(\bar{p})\gamma_5 \,g_{10}
  \notag \\
  &+ \frac{\ri k_\mu}{2m}\Lambda_+(\bar{p}) \,g_{11} 
  + \frac{k_\mu}{2m}\Lambda_+(\bar{p})\gamma_5 \,g_{12}  
  \notag \\
  &+ \gamma_\mu\Lambda_-(p) \,g_{13} 
  + \gamma_\mu\gamma_5\Lambda_-(p) \,g_{14} 
  + \frac{\bar{k}_\mu}{2m}\Lambda_-(p) \,g_{15}
  + \frac{\ri \bar{k}_\mu}{2m}\gamma_5\Lambda_-(p) \,g_{16}
  \notag \\
  &+ \frac{\ri k_\mu}{2m}\Lambda_-(p) \,g_{17} 
  + \frac{k_\mu}{2m}\gamma_5\Lambda_-(p) \,g_{18} 
  \notag \\
  &+ \Lambda_+(\bar{p})\gamma_\mu\Lambda_-(p) \,g_{19}
  + \Lambda_+(\bar{p})\gamma_\mu\gamma_5\Lambda_-(p) \,g_{20}
  \notag \\
  &+ \frac{\bar{k}_\mu}{2m}\Lambda_+(\bar{p})\Lambda_-(p) \,g_{21}
  + \frac{\ri \bar{k}_\mu}{2m}\Lambda_+(\bar{p})\gamma_5\Lambda_-(p) \,g_{22}
  \notag \\
  &+ \frac{\ri k_\mu}{2m}\Lambda_+(\bar{p})\Lambda_-(p) \,g_{23}
   + \frac{k_\mu}{2m}\Lambda_+(\bar{p})\gamma_5\Lambda_-(p) \,g_{24}\bigg],
\end{align}

\begin{table}[h!]
\centering{
\begin{tabular}{l|c|c|c|c|c}
   \hl Space--time symmetry & \wn$\mc{P}$\wn & \wn$\mc{C}$\wn &\wn$\mc{T}$\wn & \;$\mc{CP}$\; & \,$\mc{CPT}$\, \\
   \hline\hline
   \hl $g_1,g_3,g_7,g_{9},g_{13},g_{15},g_{19},g_{21}$ & \cmark   & \cmark   & \cmark   & \cmark    & \cmark     \\      
   \hl $g_2,g_8,g_{14},g_{20}$                   & \xmark   & \xmark   & \cmark   & \cmark    & \cmark     \\      
   \hl $g_4,g_{10},g_{16},g_{22}$                & \xmark   & \cmark   & \xmark   & \xmark    & \cmark     \\      
   \hl $g_5,g_{11},g_{17},g_{23}$                & \cmark   & \xmark   & \xmark   & \xmark    & \cmark     \\      
   \hl $g_6,g_{12},g_{18},g_{24}$                & \xmark   & \xmark   & \cmark   & \cmark    & \cmark     \\      
\end{tabular}
}
\caption{Summery of the preserved/violated space--time symmetries by the covariant structures associated with each form factor $g_i$.
The symbols \cmark\, and \xmark\, indicate if the space--time symmetry is preserved or violated, respectively.}
\end{table}


- Give the from-factor decomposition in the standard kinetical configuration and in ``ours''.

- Remark that the decomposition is gauge dependent (Ward identities).

\subsubsection{Extracting the contribution from $a_\tau$}
\label{sec:a_tau_term}

- Write matrix element with the general vertex decomposition.

- Perform an expansion of the form factors in $\alpha$.

- Remark again the presence of gauge dependence in this decomposition.

- Remark the absence of effects due to real radiation in the form factor approach.

- Perform Taylor expansion of the form factors around $\frac{x}{s} \sim \frac{m^2}{s}$ ($x=t,u$) to extract $a_\tau$ and $d_\tau$ contributions.

\subsubsection{Coulomb term and Sommerfeld enhancement}
\label{sec:Coulomb_term}

- Explain the need of including the Coulomb term.

- Give the contribution from Coulomb term.

\subsubsection{Analytical expressions for the helicity amplitudes (?)}
\label{sec:analytics}

- Give analytical expression in the spinor-helicity formalism for $\M^{(0)}$, $\M^{a_\tau}$, $\M^{d_\tau}$.

\subsection{Technical aspects of the calculation and employed tools}
\label{sec:tools}

- Comment the use of equivalent-photon approximation and \gammaUPC.

- Comment the use of the ChFF to parameterize the photon flux.

- Comment the use of iNWA to include spin correlations.

- Comment the use of mixed input-parameter scheme.



\subsubsection*{Remarks about the NLO prediction (?)}

- Reference to the previous paper for details on the NLO corrections.

- Mention the gauge invariant splitting of the NLO corrections.


\subsubsection*{Checks on the calculation (?)}
\label{sec:checks}

- Analytical expressions checked against FeynArts.

- Correct behaviour under $\mc{P}$, $\mc{C}$ and $\mc{CP}$ of $\M^{(0)}$, $\M^{a_\tau}$, $\M^{d_\tau}$  (?).

\subsection{Set-up of the calculation}
\label{sec:setup}

\subsubsection*{Numerical input}
\label{sec:inputs}

- Numerical inputs

\subsubsection*{Event selection (?)}
\label{sec:selection}

- ATLAS set-up

\clearpage
\newpage
\section{Results}
\label{sec:results}

%\JH{Increase labels in plots.}


\subsection{Determination of $a_\tau$ in UPCs}
\label{sec:a_tau_determination}

\begin{figure}
    \centering{
    \raisebox{0pt}{\includegraphics[width=0.5\columnwidth]{./Plots/UPC_FF_vs_FO_chff_5.02e3_cts/pt_lm_exp_vs_NLO.pdf}}
    \hspace{-0.3cm}
    \raisebox{0pt}{\includegraphics[width=0.5\columnwidth]{./Plots/UPC_FF_vs_FO_chff_5.02e3_cts/abs_eta_lm_exp_vs_NLO.pdf}}
    \newline
    (a) \hspace{7.3cm} (b) \hspace{-5cm}
    \newline   
    \raisebox{0pt}{\includegraphics[width=0.5\columnwidth]{./Plots/UPC_FF_vs_FO_chff_5.02e3_cts/acoplanarity_lmlp_FF_vs_NLO.pdf}}
    \hspace{-0.3cm}
    \raisebox{0pt}{\includegraphics[width=0.5\columnwidth]{./Plots/UPC_FF_vs_FO_chff_5.02e3_cts/cos_th_lmlp_FF_vs_NLO.pdf}}
    \newline
    (c) \hspace{7.15cm} (d) \hspace{-1.4cm}
    }
    \caption{Predictions for $\processmue$ induced by UPCs of two lead ions with $\sqrt{s_{\mr{PbPb}}}=5.02 \TeV$.
    The panels (a) and (b) show predictions for the transverse momentum and the pseudorapidity of $\mu^-$, and
    the panels (c) and (d) provide predictions for the acoplanarity and the angle between $\mu^-$ and $e^+$, respectively.
    \change{The different curves show $\ldots$.}
    The lower panel gives the relative correction $\delta^i = \frac{\Delta\sigma^i}{\sigma^\LO}$. \\
    \JH{Show plots like the upper panels or like the lower panels?.
    I prefer the upper ones, but maybe for less $a_\tau$-values ($a_\tau = -0.057, -0.003, 0.001, 0.024$) and remove QED (prd).}
    }
    \label{fig:UPC_chff}
\end{figure}

- Show the comparison between $\sigma^\mr{NLO}$ vs $\sigma^{\mr{FF}}$ in the ATLAS set-up.

- Remark the importance of including corrections to the $\tau$-decays in the form-factor calculation.
Specially if $p_{\mr{T},\mu}$ is employed.  

- Recall the importance of the photon-flux parametrization. 

- Recall the importance of including spin correlations between the produced $\tau$-leptons. 

\clearpage 
\newpage

%\subsection{Comparison between the form-factor and the fixed-order calculation}
\subsection{Validity of the form-factor approach}
\label{sec:FF_validity}

- Show the relevance of the non-included effects in the form-factor approach, \ie
%
\begin{align}
  \Delta_\mr{FF}
  = \frac{\sigma^{\mr{FF}} - \sigma^{\mr{QED}}_\mr{P}}{\sigma^{\mr{QED}}_\mr{P}}
\end{align}

- Show $\frac{t-m^2}{s}$ (?) at LO (?) at NLO (?).


\subsubsection{Inclusive $\tau$-pair production}
\label{sec:aa_to_tautau}

- Comparison just for the production, \ie no $\tau$-decays.

- Show it just for unpolarized $\tau$-leptons.

- Remark that $a_\tau$ is part of QED corrections, which are gauge invariant independently and, thus, they can be compared.


\subsubsection*{Fixed photon--photon centre-of-mass energy}

- Show comparison for fixed c.m.e. ($3.8, 5, 10, 18 \GeV$? or all?)

- Remark importance of the Coulomb term for small c.m.e..

- Remark the importance of using observables that are not sensitive to collinear radiation off the produced $\tau$-leptons.

\begin{table}
  \centering{
  \begin{tabular}{c||c|c||c|c|c||c}
  \hl $\sqrt{s_{\gamma\gamma}}\,[\mr{GeV}]$ & $\sigma^\LO\,[\mr{nb}]$  & $\delta^{\mr{QED}}\,[\%]$  & $\delta^{a_\tau}\,[\%]$ & $\delta^{\mr{Coul.}}\,[\%]$ & $\delta^{\mr{FF}}\,[\%]$ & $\Delta_\mr{FF} \, [\%]$ \\
  \hline \hline
  \hl $3.8$                                 & $7.9548(1)$              & $2.6853(4)$                & $0.3901(1)$             & $2.4757(1)$                 & $2.8658(1)$              & \wm$0.1804(4)$ \\   
  \hline
  \hl $5$                                   & $14.065(1)$              & $0.7933(1)$                & $0.3009(1)$             & $0.4158(1)$                 & $0.7167(1)$              & $-0.0765(1)$   \\   
  \hline
  \hl $10$                                  & $7.1323(1)$              & $0.3857(1)$                & $0.2876(1)$             & $0.0196(1)$                 & $0.3072(1)$              & $-0.0785(1)$   \\   
  \hline
  \hl $18$                                  & $3.0310(1)$              & $0.6996(2)$                & $0.2842(1)$             & $0.0018(1)$                 & $0.2856(1)$              & $-0.4136(2)$
  \end{tabular}
  }
  \caption{
  Inclusive cross-section for $\gamma\gamma\to\tau^+\tau^-$ for different fixed photon--photon centre-of-mass energies, which are indicated in the first column. 
  The second column gives the LO cross section.
  The third column provides the QED relative correction.
  The fourth and fifth columns correspond to the relative corrections from the anomalous magnetic moment and from the Coulomb term, respectively.
  The sixth column gives the total relative correction in the form-factor approach.
  Finally, the last column correspond to the difference between the NLO QED relative correction and the form-factor relative correction.
  }
  \label{tbl:prd_cme}
\end{table}


\clearpage
\newpage

\begin{figure}
    \centering{
    \raisebox{0pt}{\includegraphics[width=0.5\columnwidth]{./Plots/PRD_FF_vs_FO_tau_cme_3.8/pt_Lm.pdf}}
    \hspace{-0.3cm}
    \raisebox{0pt}{\includegraphics[width=0.5\columnwidth]{./Plots/PRD_FF_vs_FO_tau_cme_5/pt_Lm.pdf}}
    \newline
    (a) \hspace{7.3cm} (b) \hspace{-5cm}
    \newline   
    \raisebox{0pt}{\includegraphics[width=0.5\columnwidth]{./Plots/PRD_FF_vs_FO_tau_cme_10/pt_Lm.pdf}}
    \hspace{-0.3cm}
    \raisebox{0pt}{\includegraphics[width=0.5\columnwidth]{./Plots/PRD_FF_vs_FO_tau_cme_18/pt_Lm.pdf}}
    \newline
    (c) \hspace{7.15cm} (d) \hspace{-1.4cm}
    }
    \caption{Predictions for $\gamma\gamma\to\tau^+\tau^-$ for different photon--photon centre-of-mass energies:
    (a) $\sqrt{s_{\gamma\gamma}} = 3.8\,\GeV$, 
    (b) $\sqrt{s_{\gamma\gamma}} = 5\,\GeV$,
    (c) $\sqrt{s_{\gamma\gamma}} = 10\,\GeV$, 
    and (d) $\sqrt{s_{\gamma\gamma}} = 18\,\GeV$.
    The different curves show the transverse momentum of the $\tau^-$-lepton at LO (black) and including NLO QED correction (red),
    the contribution from the anomalous magnetic moment (green),
    the contribution from the Coulomb term (blue),
    and the total correction obtained in form-factor approach (brown).
    The lower panel gives the relative correction $\delta^i = \frac{\Delta\sigma^i}{\sigma^\LO}$. \\
    \JH{Remove (unpol.) in title and add subscript $\gamma\gamma$.}
    }
    \label{fig:PRD_cme_pT_Lm}
\end{figure}
%
\begin{figure}
    \centering{
    \raisebox{0pt}{\includegraphics[width=0.5\columnwidth]{./Plots/PRD_FF_vs_FO_tau_cme_3.8/eta_Lm.pdf}}
    \hspace{-0.3cm}
    \raisebox{0pt}{\includegraphics[width=0.5\columnwidth]{./Plots/PRD_FF_vs_FO_tau_cme_5/eta_Lm.pdf}}
    \newline
    (a) \hspace{7.3cm} (b) \hspace{-5cm}
    \newline   
    \raisebox{0pt}{\includegraphics[width=0.5\columnwidth]{./Plots/PRD_FF_vs_FO_tau_cme_10/eta_Lm.pdf}}
    \hspace{-0.3cm}
    \raisebox{0pt}{\includegraphics[width=0.5\columnwidth]{./Plots/PRD_FF_vs_FO_tau_cme_18/eta_Lm.pdf}}
    \newline
    (c) \hspace{7.15cm} (d) \hspace{-1.4cm}
    }
    \caption{Same as Fig.~\ref{fig:PRD_cme_pT_Lm}, but for the pseudorapidity of the $\tau^-$-lepton.\\
    \JH{Change to $|\eta_\tau|$ and more bins.} \\
    \JH{Remove (unpol.) in title and add subscript $\gamma\gamma$.}
    }
    \label{fig:PRD_cme_eta_Lm}
\end{figure}



\clearpage
\newpage
\subsubsection*{Ultraperipheral heavy-ion collision}

\begin{table}
  \centering{
  \begin{tabular}{c|c||c|c|c||c}
  \hl $\sigma^\LO\,[\mr{mb}]$  & $\delta^{\mr{QED}}\,[\%]$  & $\delta^{a_\tau}\,[\%]$ & $\delta^{\mr{Coul.}}\,[\%]$ & $\delta^{\mr{FF}}\,[\%]$ & $\Delta_\mr{FF} \, [\%]$ \\
  \hline \hline
  $1.0617(1)$              & $0.945(1)$                 & $0.308(1)$              & $0.589(1)$                  & $0.898(1)$               & $-0.047(1)$      
  \end{tabular}
  }
  \caption{
  Inclusive cross-section for $\gamma\gamma\to\tau^+\tau^-$ induced by UPCs of two lead ions with $\sqrt{s_{\text{PbPb}}}=5.02 \TeV$. 
  The first column gives the LO cross section.
  The second column provides the QED relative correction.
  The third and fourth columns correspond to the relative corrections from the anomalous magnetic moment and from the Coulomb term, respectively.
  The fifth column gives the total relative correction in the form-factor approach.
  Finally, the last column correspond to the difference between the NLO QED relative correction and the form-factor relative correction.
  }
  \label{tbl:prd_chff}
\end{table}

- Show comparison with photon flux.

- Mention the problem in Ref.~\cite{Jiang:2024dhf}: 

 * Wrong input-parameter scheme $\to$ Fixable by considering the cross section, not just the correction. 

 * They do not include the Coulomb term $\to$ Wrong value for $a_\tau$ obtained. 

\begin{figure}
    \centering{
    \raisebox{0pt}{\includegraphics[width=0.5\columnwidth]{./Plots/PRD_FF_vs_FO_tau_chff_5.02e3/pt_Lm.pdf}}
    \hspace{-0.3cm}
    \raisebox{0pt}{\includegraphics[width=0.5\columnwidth]{./Plots/PRD_FF_vs_FO_tau_chff_5.02e3/eta_Lm.pdf}}
    \newline
    (a) \hspace{7.15cm} (b) \hspace{-1.4cm}
    }
    \caption{Inclusive cross-section for $\gamma\gamma\to\tau^+\tau^-$ induced by UPCs of two lead ions with $\sqrt{s_{\mr{PbPb}}}=5.02 \TeV$. 
    The panels (a) and (b) show the prediction for the transverse momentum and the pseudorapidity of the $\tau^-$-lepton, respectively.
    The different curves show prediction at LO (black) and including NLO QED correction (red),
    the contribution from the anomalous magnetic moment (green),
    the contribution from the Coulomb term (blue),
    and the total correction obtained in form-factor approach (brown).
    The lower panel gives the relative correction $\delta^i = \frac{\Delta\sigma^i}{\sigma^\LO}$.\\
    \JH{Change (b) to $|\eta_\tau|$ and more bins.} \\
    \JH{Remove title.}
    }
    \label{fig:PRD_chff}
\end{figure}


\newpage

\subsubsection{$\tau$-pair production in UPCs assuming leptonic $\tau$-decays}

- Comparison including leptonic $\tau$-decays.

- Show the comparison between $\sigma^\mr{QED}_\mr{P}$ vs $\sigma^{\mr{FF}}$.

- Remark that the corrections to the production are gauge invariant by their own and, thus, can be compared to determine $\Delta_\mr{FF}$.


\begin{table}
  \centering{
  \begin{tabular}{c||c|c||c|c|c||c}
  \hl cuts                                             
  & $\sigma^\LO\,[\mr{nb}]$
  & $\delta^{\mr{QED}}\,[\%]$   & $\delta^{a_\tau}\,[\%]$ & $\delta^{\mr{Coul.}}\,[\%]$ & $\delta^{\mr{FF}}\,[\%]$ & $\Delta_{\mr{FF}}[\%]$ \\
  \hline \hline
  \Hl Inclusive   
  & $\mwn32920(3)$
  & $0.9440(7)$               & $0.3084(1)$             & $0.5893(1)$                 & $0.8976(1)$              & $-0.046(1)$    \\   
  \hline
  \hl $\begin{matrix}p_{\mr{T},\ell}>4\,\mr{GeV}\\ |\eta_\ell|<2.5 \mwm\mwm\end{matrix}$   
  & $\mwn45.877(7)$ 
  & $0.3792(8)$               & $0.3483(1)$             & $0.0017(1)$                 & $0.3500(1)$              & $-0.029(1)$    \\   
  \hline
  \hl $\begin{matrix}p_{\mr{T},\ell}>2\,\mr{GeV}\\ |\eta_\ell|<2.5 \mwm\mwm\end{matrix}$   
  & $438.22(6)\mwn$  
  & $0.3021(5)$               & $0.3310(1)$             & $0.0180(1)$                 & $0.3490(1)$              & $\mwm0.047(1)$ \\   
  \hline
  \hl $\begin{matrix}p_{\mr{T},\ell}>6\,\mr{GeV}\\ |\eta_\ell|<2.5 \mwm\mwm\end{matrix}$   
  & $\mwn10.137(2)$ 
  & $0.456(2)\mwn$            & $0.3531(1)$             & $0.0004(1)$                 & $0.3535(1)$              & $-0.102(1)$ \\   
  \hline
  \hl $\begin{matrix}p_{\mr{T},\ell}>4\,\mr{GeV}\\ |\eta_\ell|<3.5 \mwm\mwm\end{matrix}$   
  & $\mwn49.911(8)$ 
  & $0.3845(8)$               & $0.3445(1)$             & $0.0017(1)$                 & $0.3463(1)$              & $-0.038(1)$    \\
  \hline
  \hl $\begin{matrix}p_{\mr{T},\ell}>4\,\mr{GeV}\\ |\eta_\ell|<2 \mwm\mwm\mwm\end{matrix}$ 
  & $\mwn39.237(6)$
  & $0.371(1)\mwn$            & $0.3532(1)$             & $0.0017(1)$                 & $0.3549(1)$              & $-0.016(1)$
  \end{tabular}
  }
  \caption{
  Cross section for  $\gamma\gamma\to\tau^+\tau^-\to e^+\mu^-\bar{\nu}_\tau\nu_\tau\bar{\nu}_\mu\nu_e$ induced by UPCs of two lead ions with $\sqrt{s_{\mr{PbPb}}}=5.02 \TeV$ for different cut configurations. \\
  \JH{Change.}
  }
  \label{tbl:upc_chff}
\end{table}


\begin{comment}
\subsubsection*{Fixed photon--photon centre-of-mass energy (?)}

- Show inclusive predictions for fixed c.m.e.

- Remark again the dominance of the Coulomb term in the NLO correction for small c.m.e. 

- Show that the sensitivity of $p_{\mr{T},\mu}$ to collinear radiation off the produced $\tau$-leptons is not determinant and, thus, it is a good observable.
\end{comment}

\clearpage
\newpage

%\subsubsection*{ATLAS set-up}

%- Show the comparison for the ATLAS set-up.

%- Comment that the Coulomb contribution is suppressed (neglectable) here due to the presence of $p_{\mr{T},\ell}$ cuts.




\subsubsection*{Dependence on the choice of $p_{\mr{T},\ell}$ cuts}

\begin{figure}
    \centering{
    \raisebox{0pt}{\includegraphics[width=0.5\columnwidth]{./Plots/UPC_FF_vs_FO_chff_5.02e3/pt_lm_FF_vs_QED.pdf}}
    \hspace{-0.3cm}
    \raisebox{0pt}{\includegraphics[width=0.5\columnwidth]{./Plots/UPC_FF_vs_FO_chff_5.02e3/abs_eta_lm_FF_vs_QED.pdf}}
    \newline
    (a) \hspace{7.3cm} (b) \hspace{-5cm}
    \newline   
    \raisebox{0pt}{\includegraphics[width=0.5\columnwidth]{./Plots/UPC_FF_vs_FO_chff_5.02e3/acoplanarity_lmlp_FF_vs_QED.pdf}}
    \hspace{-0.3cm}
    \raisebox{0pt}{\includegraphics[width=0.5\columnwidth]{./Plots/UPC_FF_vs_FO_chff_5.02e3/cos_th_lmlp_FF_vs_QED.pdf}}
    \newline
    (c) \hspace{7.15cm} (d) \hspace{-1.4cm}
    }
    \caption{Predictions for $\processmue$ induced by UPCs of two lead ions with $\sqrt{s_{\mr{PbPb}}}=5.02 \TeV$.
    The panels (a) and (b) show predictions for the transverse momentum and the pseudorapidity of $\mu^-$, and
    the panels (c) and (d) provide predictions for the acoplanarity and the angle between $\mu^-$ and $e^+$, respectively.
    The different curves show the transverse momentum of the $\tau^-$-lepton at LO (black) and including NLO QED correction (red),
    the contribution from the anomalous magnetic moment (green),
    the contribution from the Coulomb term (blue),
    and the total correction obtained in form-factor approach (brown).
    The lower panel gives the relative correction $\delta^i = \frac{\Delta\sigma^i}{\sigma^\LO}$. \\
    \JH{Remove title.}
    }
    \label{fig:UPC_chff}
\end{figure}

- Show the comparison for $p_{\mr{T},\ell} > 2, 4, 6 \GeV$ and $|\eta_\ell| < 2.5$.

- Add $p_{\mr{T},\ell} > 0 \GeV$ and $|\eta_\ell| < 2.5$ (?).

- Comment the problem of having a large $p_{\mr{T},\ell}$ cut if enough precision is reached.

- For the plots: Do a figure for each observable and different cuts (inclusive apart) or do a figure for each cut configuration (?). 

%\subsubsection*{Comparison between the form-factor approach and the full NLO prediction}

%- Show $\sigma^\mr{FF}$ vs. $\sigma^\NLO$ including corrections to the decays.

%- Remark the importance of including the corrections to the decays in the form-factor calculation (or of extracting these corrections to the fixed-order calculation).
%Specially if $p_{\mr{T},\mu}$ is employed in the analysis.  

\subsubsection*{Spin-correlation effects (?)}

- I would say no, it is repetitive with the paper. 
I would just mention it in Sec.~\ref{sec:a_tau_determination}.





%\subsection{Impact of BSM contributions to $a_\tau$ in the cross section}

%- Show plot with different values of $a_\tau$. (Maybe use $a_\tau^\mr{SM} + a_\tau^\mr{BSM}$).

\subsection{Impact of a non-zero electric dipole moment in the cross section (?)}

- $d_\tau^\mr{SM} = 0$.

- $\order{d_\tau^\mr{BSM}} \to 0$.

- $\order{(d_\tau^\mr{BSM})^2}$ terms needed.

\section{Conclusion}
\label{sec:conclusion}



\appendix
\section*{Appendix}



\newpage

\bibliographystyle{utphys.bst}
\bibliography{aa-tautau}


\end{document}



